\section{Personas} \label{sec:personas}

\begin{table}[h!]
\centering
\begin{tabular}{|p{0.08\linewidth}|p{0.07\linewidth}|p{0.37\linewidth}|p{0.37\linewidth}|}
\hline
\textbf{Name} & \textbf{Alter} & \textbf{Vorwissen} & \textbf{Ziel} \\ \hline

Anna & 18 & War schon oft als Kind bei diesem Zahnarzt; Hat noch nie selber einen Termin beim Zahnarzt ausgemacht; Hat keine Ahnung über angebotene Leistungen & Informieren über alle angebotenen Leistungen und Buchen eines Termines über das Web \\ \hline

Robert & 25 & Vor einem Monat umgezogen; Weiß genau, welche Leistungen er benötigt & Informieren über die Qualität der Praxis und des Teams; Standort und Anfahrt herausfinden; Termin über E-Mail vereinbaren und Fragen stellen \\ \hline

Hans & 65 & Langjähriger Kunde; Unerfahren mit Computern & Muss Telefonnummer zur Terminvereinbarung finden und anrufen \\ \hline

Kilian & 23 & Ist ein ausgebildeter Zahnarzthelfer; Ist derzeit auf Suche nach einer neuen Arbeitsstelle & Informieren über die Praxis und das Team; Standort und Anfahrt herausfinden; Offene Stellen ansehen; Kontaktmöglichkeiten für Bewerber \\ \hline

\end{tabular}
\caption{Personas}
\label{tab:personas}
\end{table}

Wie in Tabelle \ref{tab:personas} dargestellt, ist das Ziel für einen typischen Nutzer, möglichst schnell die für ihn relevanten Informationen zu finden. Basierend auf diesen Informationen möchte er immer einen Kontakt herstellen - sei es über das angebotene Formular zur Terminvereinbarung, per E-Mail oder telefonisch. Dabei ist der erste Eindruck maßgeblich, um das Vertrauen der Nutzer zu gewinnen.