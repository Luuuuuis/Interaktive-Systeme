\section{Heuristische Evaluation} \label{sec:evulation}

Bei der Erstellung von Software ist es essenziell, dass die Benutzeroberfläche ergonomisch gestaltet ist und die Benutzer- und Aufgabenorientiertheit berücksichtigt wird. Die Anforderungen an die Software in der Interaktion mit dem Menschen sind in der Normenreihe DIN EN ISO 9241 "Ergonomie der Mensch-System-Interaktion" festgelegt. 

Die "charakteristischen Eigenschaften dargestellter Informationen" aus Teil 12 (\ref{sec:informationsdarstellung}) unterstützen die "Grundsätze der Dialoggestaltung" aus Teil 110 (\ref{sec:dialoggestaltung}), die wiederum die "Anforderungen an die Gebrauchstauglichkeit" aus Teil 11 (\ref{sec:usability}) unterstützen. Sie hängen also zusammen und beeinflussen, wie effektiv, effizient und zufriedenstellend die typischen Personas aus Kapitel \ref{sec:personas} ihr Ziel erreichen können.

Anhand dieser Richtlinien und Kriterien kann eine heuristische Evaluation durchgeführt werden.

\subsection{Ergonomische Informationsdarstellung} \label{sec:informationsdarstellung}

Die Norm DIN EN ISO 9241 Teil 12 legt die visuellen Eigenschaften der dargestellten Informationen in einem System fest. Die Grundsätze der grafischen Gestaltung von Informationen werden anhand der folgenden Kriterien an der Webseite untersucht:

Die \textbf{Lesbarkeit} von Schriften ist sehr wichtig, da auf dieser Webseite Text das prominenteste Mittel zur Informationsdarstellung ist. Die gewählte Schriftart ist "Lora-Regular". Diese Schriftart verfügt über Serifen, was die Lesbarkeit von kleinen Texten beeinträchtigt. Die kleine Schriftgröße von 16 Pixel ist auf kleinen Bildschirmen schwer lesbar. Die Schriftlaufweite - der Zeichenabstand mehrerer Zeichen - ist negativ, was zu überschneiden Buchstaben führt. Dieses Phänomen kann man beispielsweise bei der Kombination von "Te" beobachten und erschwert das Lesen weiter. Darüber hinaus ist die sekundäre Schriftfarbe von Hellgrau auf weißem Hintergrund schwer zu lesen und sollte zu Schwarz geändert werden. Eine weitere Möglichkeit primären Text hervorzuheben, ist die Verwendung von Fettschrift.

Die \textbf{Konsistenz} von Fluchtlinien ist auf der Webseite gegeben. Elemente wurden mit Abstand an den Rändern ausgerichtet und die Abstände zwischen Elementen sind durchgängig. Alleinig eine Hervorhebung durch einen grauen Kasten im Titelbild auf jeder Seite bricht diese Konvention. Auch die Farben sind konsistent und repräsentieren ein professionelles Corporate Design. Sie werden sparsam eingesetzt und die Farbpalette ist auf vier Farben beschränkt.

Die \textbf{Klarheit} der dargestellten Informationen auf der Webseite ist gegeben. Die Informationen sind in Blöcken gruppiert und die wichtigsten Daten sind dabei im Vordergrund platziert. Direkt an erster Stelle auf der Startseite befindet sich der Block mit den Sprechzeiten, Kontaktmöglichkeiten und der online Terminvereinbarung. Die Anfahrt bzw. der Standort ist durch eine Einbindung von Google Maps gut ersichtlich. Auch befindet sich auf der Startseite die Möglichkeit, direkt über die Top-Navigation zu den Unterseiten zu springen, um genau die Informationen zu finden, die der Nutzer sucht. Alle Unterseiten sind mit relevanten Inhalten bestückt. Lediglich die Unterseite "Praxis" ist ausschließlich mit aussagelosen Bildern bestückt und sollte mit mehr Informationen ergänzt werden, z.B. zu Parkplätzen oder Einrichtungen für Menschen mit Behinderung.

Der Nutzer erhält bei weiterem Scrollen Zusammenfassungen der Unterseiten und kann per Link dorthin springen. Das reduziert jedoch die \textbf{Kompaktheit} der Seite und birgt die Gefahr von Doppelungen. Die Startseite wirkt dadurch unübersichtlich und teilweise redundant. Alleinig wichtige Informationen - wie Stellenausschreibungen, Anfahrt und Kontakt - sollten erneut kompakt hervorgehoben werden. Die Patientenstimmen sind zu lang und gehen in der Masse unter. Es wird empfohlen, diese auf ein Minimum reduzieren und stattdessen durch Sterne oder Icons zu ersetzen.

Damit die \textbf{Unterscheidbarkeit} und \textbf{Erkennbarkeit} der Heraushebungen gewährleistet wird, wird auf dieser Webseite ein klar abhebender Kasten verwendet. Dazu könnten noch weitere Icons für eine schnellere und einfachere Erkennung eingesetzt werden. Frei liegender Text ohne Kasten - wie auf der Startseite über "Unsere Leistungen" - wird wegen mangelnder Unterscheidbarkeit übersehen.
\subsection{Ergonomische Dialoggestaltung} \label{sec:dialoggestaltung}

Die Norm DIN EN ISO 9241 Teil 110 legt Grundsätze der Dialoggestaltung zwischen Mensch und Software fest. Danach wird im folgenden die Webseite analysiert:

Die \textbf{Aufgabenangemessenheit} untersucht die geeignete Funktionalität von Elementen, sodass Interaktionen auf einem Minimum gehalten werden können. Dies geschieht durch die Verwendung von sogenannten "Defaults" (Standards). Ein gutes Beispiel dafür ist das Terminbuchungsformular unter den Knöpfen "Online-Termine". In den Drop-Down-Menüs sind Standardwerte eingetragen, die für die meisten Nutzer zutreffen. Sie müssen nur bei Bedarf geändert werden. Für den Nutzer ist lediglich ein Klick auf den angezeigten Termin erforderlich. Die Gestaltung der Webseite ist anthropozentrisch auf die Bedürfnisse von Zahnärzten ausgerichtet.

Die \textbf{Selbstbeschreibungsfähigkeit} zeigt sich in dem Kontaktformular an einem positiven und einem negativen Beispiel. Positiv ist, dass durch eine klare Kennzeichnung von Pflichtfeldern mittels Sternchen dem Nutzer bereits beim Ausfüllen klar ist, was er ausfüllen muss. Zudem wird eine Fehlermeldung angezeigt, wenn der Nutzer versucht, das Formular mit einem leeren Pflichtfeld abzusenden. Allerdings fällt negativ auf, dass der "Senden"-Knopf bei einem Mouse-Over deaktivert ist und nicht anklickbar ist, wenn die Checkbox zum Datenschutz noch nicht bestätigt wurde. Dabei kriegt der Nutzer keinen Hinweis, wieso er das Formular nicht absenden kann. Darüber hinaus ist die Checkbox nicht als Pflichtfeld gekennzeichnet.

Die \textbf{Erwartungskonformität} stellt sich durch die Konsistenz und das Vorwissen von anderen Webseiten dar. Die Top-Navigation Bar entspricht den gängigen Standards und das Logo ist links platziert, sodass mit einem Klick auf die Startseite zurücknavigiert wird. Die Navigation ist auf jeder Unterseite und an jedem Scroll-Punkt konsistent.Nur das Popover des Terminbuchungsformulars ist eine Aussnahme. Die aktuell angezeigte Seite wird durch eine farbliche Hervorhebung in der Navigation gekennzeichnet, was eine übersichtliche und schnelle Navigation ermöglicht. Allerdings sind nicht alle Knöpfe in der Farbe und im Mouse-Over konsistent: Manche sind schwarz mit weißem Text, andere weiß mit schwarzem Text. Manche Knöpfe haben einen Mouse-Over-Effekt (Online-Termine), andere nicht (Mehr erfahren). Ebenso sind Links nicht eindeutig als solche gekennzeichnet und haben unterschiedliche Farben, insbesondere E-Mail-Adressen und im Footer. Die Tabwege sind auf der Startseite sowie im Terminbuchungsformular kurz und erwartungskonform. Alle Navigationselemente und Verlinkungen werden von links nach rechts und von oben nach unten durchgegangen.
\subsection{Gebrauchstauglichkeit (Usability)} \label{sec:usability}

In der Norm DIN EN ISO 9241 Teil 11 werden die Anforderungen an die Gebrauchstauglichkeit von Software in der Interaktion mit dem Menschen festgelegt. Usability ist das Produkt aus Effektivität, Effizienz und Zufriedenheit. Dieses Produkt ergibt sich aus der Informationsdarstellung und der Dialoggestaltung. 

Die \textbf{Effektivität} der gegebenen Webseite für die oben definierten Personas ist gegeben, da die Nutzer ihre Ziele erreichen können. Das Vereinbaren eines Termins per Online-Formular ist einfach zu finden und zu bedienen. Auch die Telefonnummer und E-Mail-Adresse sind weitestgehend leicht zu finden und anzuklicken. Alle relevanten Informationen sind auf der Webseite vorhanden und durch die Top-Navigation leicht zu finden. Offene Stellenanzeigen bzw. Karrieremöglichkeiten sind klar gekennzeichnet und wirken durch eine Hervorhebung auf der Startseite als einladend und erwünscht.

Die \textbf{Effizienz} ist ebenfalls gegeben, da die Nutzer ihre Ziele schnell und mit wenig Aufwand erreichen können. Gut strukturierte Unterseiten und eine klare Navigation mit kurzen Tabwegen ermöglichen eine schnelle und effiziente Interaktion. Des Weiteren sind Formulare mit Standardwerten vorausgefüllt und die sinnvolle Anordnung der Eingabeelemente reduziert die Interaktion auf ein Minimum. Lediglich die Schriftart und Schriftfarbe beeinträchtigen ein schnelles Überfliegen der Seite.

Die \textbf{Zufriedenheit} setzt eine die effektive und effiziente Interaktion vorraus. Das Vertrauen des Nutzers wird durch eine professionelle und benutzerfreundliche Gestaltung der Webseite gewonnen. Der Nutzer verlässt die Webseite mit dem Erreichen seiner Ziele und einem positiven Gefühl. Die Aufteilung der Seite ist erwartungsgemäß des Internetstandards und eine Nutzung ist ohne Lernaufwand möglich. 
