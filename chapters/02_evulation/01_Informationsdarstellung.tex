\subsection{Ergonomische Informationsdarstellung} \label{sec:informationsdarstellung}

Die Norm DIN EN ISO 9241 Teil 12 legt die visuellen Eigenschaften der dargestellten Informationen in einem System fest. Die Grundsätze der grafischen Gestaltung von Informationen werden anhand der folgenden Kriterien an der Webseite untersucht:

Die \textbf{Lesbarkeit} von Schriften ist sehr wichtig, da auf dieser Webseite Text das prominenteste Mittel zur Informationsdarstellung ist. Die gewählte Schriftart ist "Lora-Regular". Diese Schriftart verfügt über Serifen, was die Lesbarkeit von kleinen Texten beeinträchtigt. Die kleine Schriftgröße von 16 Pixel ist auf kleinen Bildschirmen schwer lesbar. Die Schriftlaufweite - der Zeichenabstand mehrerer Zeichen - ist negativ, was zu überschneiden Buchstaben führt. Dieses Phänomen kann man beispielsweise bei der Kombination von "Te" beobachten und erschwert das Lesen weiter. Darüber hinaus ist die sekundäre Schriftfarbe von Hellgrau auf weißem Hintergrund schwer zu lesen und sollte zu Schwarz geändert werden. Eine weitere Möglichkeit primären Text hervorzuheben, ist die Verwendung von Fettschrift.

Die \textbf{Konsistenz} von Fluchtlinien ist auf der Webseite gegeben. Elemente wurden mit Abstand an den Rändern ausgerichtet und die Abstände zwischen Elementen sind durchgängig. Alleinig eine Hervorhebung durch einen grauen Kasten im Titelbild auf jeder Seite bricht diese Konvention. Auch die Farben sind konsistent und repräsentieren ein professionelles Corporate Design. Sie werden sparsam eingesetzt und die Farbpalette ist auf vier Farben beschränkt.

Die \textbf{Klarheit} der dargestellten Informationen auf der Webseite ist gegeben. Die Informationen sind in Blöcken gruppiert und die wichtigsten Daten sind dabei im Vordergrund platziert. Direkt an erster Stelle auf der Startseite befindet sich der Block mit den Sprechzeiten, Kontaktmöglichkeiten und der online Terminvereinbarung. Die Anfahrt bzw. der Standort ist durch eine Einbindung von Google Maps gut ersichtlich. Auch befindet sich auf der Startseite die Möglichkeit, direkt über die Top-Navigation zu den Unterseiten zu springen, um genau die Informationen zu finden, die der Nutzer sucht. Alle Unterseiten sind mit relevanten Inhalten bestückt. Lediglich die Unterseite "Praxis" ist ausschließlich mit aussagelosen Bildern bestückt und sollte mit mehr Informationen ergänzt werden, z.B. zu Parkplätzen oder Einrichtungen für Menschen mit Behinderung.

Der Nutzer erhält bei weiterem Scrollen Zusammenfassungen der Unterseiten und kann per Link dorthin springen. Das reduziert jedoch die \textbf{Kompaktheit} der Seite und birgt die Gefahr von Doppelungen. Die Startseite wirkt dadurch unübersichtlich und teilweise redundant. Alleinig wichtige Informationen - wie Stellenausschreibungen, Anfahrt und Kontakt - sollten erneut kompakt hervorgehoben werden. Die Patientenstimmen sind zu lang und gehen in der Masse unter. Es wird empfohlen, diese auf ein Minimum reduzieren und stattdessen durch Sterne oder Icons zu ersetzen.

Damit die \textbf{Unterscheidbarkeit} und \textbf{Erkennbarkeit} der Heraushebungen gewährleistet wird, wird auf dieser Webseite ein klar abhebender Kasten verwendet. Dazu könnten noch weitere Icons für eine schnellere und einfachere Erkennung eingesetzt werden. Frei liegender Text ohne Kasten - wie auf der Startseite über "Unsere Leistungen" - wird wegen mangelnder Unterscheidbarkeit übersehen.