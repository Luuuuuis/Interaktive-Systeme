\subsection{Ergonomische Dialoggestaltung} \label{sec:dialoggestaltung}

Die Norm DIN EN ISO 9241 Teil 110 legt Grundsätze der Dialoggestaltung zwischen Mensch und Software fest. Nach ausgewählten Grundsätzen wird im Folgenden die Webseite analysiert:

Die \textbf{Aufgabenangemessenheit} untersucht geeignete Funktionalitäten von Elementen, sodass Interaktionen auf einem Minimum gehalten werden können. Dies geschieht durch die Verwendung von sogenannten "Defaults" (Standards). Ein gutes Beispiel dafür ist das Terminbuchungsformular unter den Knöpfen "Online-Termine". In den Drop-Down-Menüs sind Standardwerte eingetragen, die für die meisten Nutzer zutreffen. Sie müssen nur bei Bedarf geändert werden. Für den Nutzer ist lediglich ein Klick auf den angezeigten Termin erforderlich. Die Gestaltung der Webseite ist anthropozentrisch auf die Bedürfnisse von Zahnärzten ausgerichtet.

Die \textbf{Fehlertoleranz} zeigt sich in dem Kontaktformular an einem positiven und einem negativen Beispiel. Positiv ist, dass durch eine klare Kennzeichnung von Pflichtfeldern mittels Sternchen dem Nutzer bereits beim Ausfüllen klar ist, was er ausfüllen muss. Zudem wird eine Fehlermeldung angezeigt, wenn der Nutzer versucht, das Formular mit einem leeren Pflichtfeld abzusenden oder eine falsch formatierte E-Mail-Adresse angibt. Allerdings fällt negativ auf, dass der "Senden"-Knopf bei einem Mouse-Over deaktivert ist und nicht anklickbar ist, wenn die Checkbox zum Datenschutz noch nicht bestätigt wurde. Dabei kriegt der Nutzer keinen Hinweis, wieso er das Formular nicht absenden kann. Darüber hinaus ist die Checkbox nicht als Pflichtfeld gekennzeichnet. Der Text neben der Checkbox ist nicht anklickbar und es wird keine Interaktion durch Drücken neben der Checkbox ermöglicht.

Die \textbf{Erwartungskonformität} stellt sich durch die Konsistenz der Gestaltgesetze und durch das Vorwissen von anderen Webseiten dar. Die Top-Navigation Bar entspricht den gängigen Standards und das Logo ist links platziert, sodass mit einem Klick auf die Startseite zurücknavigiert wird. Die Navigation ist auf jeder Unterseite und an jedem Scroll-Punkt konsistent. Nur das Pop-Over des Terminbuchungsformulars ist eine Aussnahme. Die aktuell angezeigte Seite wird durch eine farbliche Hervorhebung in der Navigation gekennzeichnet, was eine übersichtliche und schnelle Navigation ermöglicht. Allerdings sind nicht alle Knöpfe in der Farbe und im Mouse-Over konsistent: Manche sind schwarz mit weißem Text, andere weiß mit schwarzem Text. Manche Knöpfe haben einen Mouse-Over-Effekt (Online-Termine), andere nicht (Mehr erfahren). Ebenso sind Links nicht eindeutig als solche gekennzeichnet und haben unterschiedliche Farben, insbesondere E-Mail-Adressen und im Footer. Die Tabwege sind auf der Startseite sowie im Terminbuchungsformular kurz und erwartungskonform. Alle Navigationselemente und Verlinkungen werden von links nach rechts und von oben nach unten durchgegangen.

Die \textbf{Steuerbarkeit} auf der Webseite ist aufgrund einer flachen Ordnerstruktur gut gestaltet. Jede Unterseite lässt sich jederzeit erreichen. Eine Ausnahme bildet das Terminbuchungsformular. Sobald eine Uhrzeit ausgewählt wurde und der Benutzer zur Registrierung weitergeleitet wurde, lässt sich das Pop-Over nicht mehr schließen. Der Schließenknopf ist nicht mehr sichtbar und es ist nicht möglich, das Pop-Over durch einen Klick außerhalb zu schließen. Dennoch ist der Dialogfluss simple und intuitiv gestaltet.

\textbf{Individualisierbar} ist die untersuchte Seite nicht. Weder die Sprache noch die Größe der Eingabefelder ist veränderbar. So wird die Barrierefreiheit eingeschränkt und macht die Nutzung der gesamten Seite für Menschen ohne Deutschkenntnisse unmöglich. 