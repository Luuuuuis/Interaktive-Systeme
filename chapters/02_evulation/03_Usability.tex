\subsection{Gebrauchstauglichkeit (Usability)} \label{sec:usability}

In der Norm DIN EN ISO 9241 Teil 11 werden die Anforderungen an die Gebrauchstauglichkeit von Software in der Interaktion mit dem Menschen festgelegt. Usability ist das Produkt aus Effektivität, Effizienz und Zufriedenheit. Dieses Produkt ergibt sich aus der Informationsdarstellung und der Dialoggestaltung. 

Die \textbf{Effektivität} der gegebenen Webseite für die oben definierten Personas ist gegeben, da die Nutzer ihre Ziele erreichen können. Das Vereinbaren eines Termins per Online-Formular ist einfach zu finden und zu bedienen. Auch die Telefonnummer und E-Mail-Adresse sind weitestgehend leicht zu finden und anzuklicken. Alle relevanten Informationen sind auf der Webseite vorhanden und durch die Top-Navigation leicht zu finden. Offene Stellenanzeigen bzw. Karrieremöglichkeiten sind klar gekennzeichnet und wirken durch eine Hervorhebung auf der Startseite als einladend und erwünscht.

Die \textbf{Effizienz} ist ebenfalls gegeben, da die Nutzer ihre Ziele schnell und mit wenig Aufwand erreichen können. Gut strukturierte Unterseiten und eine klare Navigation mit kurzen Tabwegen ermöglichen eine schnelle und effiziente Interaktion. Des Weiteren sind Formulare mit Standardwerten vorausgefüllt und die sinnvolle Anordnung der Eingabeelemente reduziert die Interaktion auf ein Minimum. Lediglich die Schriftart und Schriftfarbe beeinträchtigen ein schnelles Überfliegen der Seite.

Die \textbf{Zufriedenheit} setzt eine die effektive und effiziente Interaktion vorraus. Das Vertrauen des Nutzers wird durch eine professionelle und benutzerfreundliche Gestaltung der Webseite gewonnen. Der Nutzer verlässt die Webseite mit dem Erreichen seiner Ziele und einem positiven Gefühl. Die Aufteilung der Seite ist erwartungsgemäß des Internetstandards und eine Nutzung ist ohne Lernaufwand möglich. 
