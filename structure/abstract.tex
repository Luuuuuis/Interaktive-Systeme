\section*{Abstract}

Das Controller-Service-Repository (CSR)-Pattern ist ein Architekturmuster, das die Anwendung in drei Schichten aufteilt: Controller, Service und Repository. 
Jede Schicht hat eine bestimmte Aufgabe. 
Der Controller verarbeitet Eingaben, der Service enthält die Geschäftslogik und das Repository kümmert sich um die Datenpersistenz. 
Die Schichten interagieren miteinander, um die Funktionalität der Anwendung bereitzustellen.

Dieses Muster ähnelt dem Model-View-Controller (MVC)- und Domain-Driven Design (DDD)-Pattern, unterscheidet sich jedoch in der Art und Weise, wie die Anwendung aufgeteilt wird. 
Alle drei Muster zielen darauf ab, die Komplexität zu reduzieren und die Wartbarkeit zu verbessern.

Das CSR-Muster bietet mehrere Vorteile, darunter eine klare Trennung von Belangen, verbesserte Wartbarkeit und Erweiterbarkeit sowie eine effiziente Teststrategie. 
Es gibt jedoch auch Nachteile wie erhöhte Komplexität und Abstraktion, die bei kleinen Anwendungen oder unerfahrenen Entwicklern zu Problemen führen können.

Insgesamt ist das CSR-Pattern ein nützliches Werkzeug für die Entwicklung von Anwendungen, insbesondere im Bereich von Webservices und Microservices. 
Es bietet eine Reihe von Vorteilen, die zu besserem Code führen können. 
Bei der Entscheidung für die Verwendung des Musters sollten jedoch die potenziellen Nachteile berücksichtigt werden.